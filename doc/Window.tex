\documentclass{scrartcl}
\usepackage{hyperref}

\author{Si Jun Kwon}
\title{Window class}
\begin{document}
    \maketitle
    \tableofcontents
    \newpage
    \section{Window Class}
    \subsection{Description}
    The window class represents a window application. It is powered by \href{https://www.pygame.org/}{pygame}. The window has a tree of window elements which
    it uses to render the window elements each frame. The tree is traversed with each element receiving
    events occuring and are tagged to be rerendered when their state has changed.
    \subsection{Class fields}
    \begin{tabular}{|p{0.2\textwidth}|p{0.2\textwidth}|p{0.5\textwidth}|}
        \hline
        Field name & Type & Description\\
        \hline
        width & int & Defines the width of the window in pixels\\
        \hline
        height & int & Defines the height of the window in pixels\\
        \hline
        windowtitle & str & The title of the window(show in the bar)\\
        \hline
        event\_loop & function & Event loop of the window application. If not specified differently just calls render on its window elements.\\
        \hline
        screen & surface object & Surface object(PyGame object) which window elements are drawn on\\
        \hline
        window\_elements & window\_element& Tree of window elements which is traversed whenever the new frame starts.\\
        \hline
    \end{tabular}
    \subsection{Class methods}
    \begin{tabular}{|p{0.2\textwidth}|p{0.2\textwidth}|p{0.5\textwidth}|}
        \hline
        Method name & Parameters & Description\\
        \hline
        \_\_init\_\_ &  & Initializes the class and pygame\\
        \hline
        \_\_str\_\_ & &String representation of the window class object\\
        \hline
        \_\_repr\_\_ & & String representation of the window class object in iterable objects\\
        \hline
        run & & Runs the event loop of the application.\\
        \hline
    \end{tabular}
    \section{Window elements}
    \subsection{Description}
    Window elements is an abstract class of elements which are drawn onto the surface of the window application. They have a render 
    method and an event handler, the event handler is called whenever a frame is done and a new frame is starting to be drawn. Each Window elements
    has an array of children which are rendered from first element to last. Events are handled from last to first. Event handler can return
    True if event should not be passed to other nodes(like mouse events that has been already handled, allowing for stacked elements without
    stacked events).
    \subsection{Class fields}
    \begin{tabular}{|p{0.2\textwidth}|p{0.2\textwidth}|p{0.5\textwidth}|}
        \hline
        Field name & Type & Description\\
        \hline
        width & int & Defines the width of the window element in pixels\\
        \hline
        height & int & Defines the height of the window element in pixels\\
        \hline
        x & int & absolute x-Position of the element\\
        \hline
        y & int & absolut y-Position of the element\\
        \hline
        children & list of window elements & Children window elements of itself. Elements that are out of the drawing field of itself are clipped.\\
        \hline
        state\_changed & Bool & True if the inner state of window element has changed. Redraws itself and all its children on next render if set True.\\
        \hline
        surface & Pygame Surface & Surface on which the element draws\\
        \hline
    \end{tabular}
    \subsection{Class methods}
    \begin{tabular}{|p{0.2\textwidth}|p{0.2\textwidth}|p{0.5\textwidth}|}
        \hline
        Method name & Parameters & Description\\
        \hline
        \_\_init\_\_ &  & Initializes the class\\
        \hline
        \_\_str\_\_ & &String representation of the window class object\\
        \hline
        \_\_repr\_\_ & & String representation of the window class object in iterable objects\\
        \hline
        handle\_events & event & Handles incoming input events. event is a list of pygame events or single pygame event\\
        \hline
        render &  & renders itself and its children\\
        \hline
        update & force\_update & Rerenders the component if state\_chaged or force\_update is set to True\\
        \hline
    \end{tabular}
\end{document}
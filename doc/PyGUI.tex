\documentclass{scrartcl}
\usepackage{datetime, hyperref}

\title{PyGUI Documentation}
\author{Si Jun Kwon}
\newdate{date}{01}{09}{2020}
\date{\displaydate{date}}

\begin{document}
    \maketitle
    \newpage
    \tableofcontents
    \newpage
    \section{What is PyGUI?}
    PyGUI is a GUI library powered by \href{https://www.pygame.org/}{pygame}. PyGUI provides
    standard GUI elements which is drawn as pygame surfaces on a pygame window. The goal of this library
    is to provide a way to create a GUI app with pygame in a simple manner. PyGUI handles all the drawing,
    placings and event propagation on the pygame window. Custom elements can be created as classes that
    implement the render\_element interface. PyGUI only rerenders elements that have changed their visual 
    rendering which is defined by the programmer.
    \section{Documentation}
    \subsection{Window}
    The window class represents a window application. The window has a tree of window elements which
    it uses to render the window elements each frame. The tree is traversed with each element receiving
    events occuring and are tagged to be rerendered when their state has changed.
    \subsubsection{Class fields}
    \begin{tabular}{|p{0.2\textwidth}|p{0.2\textwidth}|p{0.5\textwidth}|}
        \hline
        Field name & Type & Description\\
        \hline
        width & int & Defines the width of the window in pixels\\
        \hline
        height & int & Defines the height of the window in pixels\\
        \hline
        windowtitle & str & The title of the window(show in the bar)\\
        \hline
        event\_loop & function & Event loop of the window application. If not specified differently just calls render on its window elements.\\
        \hline
        screen & surface object & Surface object(PyGame object) which window elements are drawn on\\
        \hline
        window\_elements & window\_element& Tree of window elements which is traversed whenever the new frame starts.\\
        \hline
    \end{tabular}
    \subsubsection{Class methods}
    \begin{tabular}{|p{0.2\textwidth}|p{0.2\textwidth}|p{0.5\textwidth}|}
        \hline
        Method name & Parameters & Description\\
        \hline
        \_\_init\_\_ &  & Initializes the class and pygame\\
        \hline
        \_\_str\_\_ & &String representation of the window class object\\
        \hline
        \_\_repr\_\_ & & String representation of the window class object in iterable objects\\
        \hline
        run & & Runs the event loop of the application.\\
        \hline
    \end{tabular}
\end{document}